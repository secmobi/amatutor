%!TEX encoding = UTF-8 Unicode
\chapter{Android安全}

\section{系统层安全机制}
\subsection{沙盒}

\subsection{文件系统}

\subsection{内存}

\section{应用层安全机制}
\subsection{权限模型}

\subsection{进程间通信}

\subsection{签名}

\subsection{数字版权管理}

\section{提权漏洞}

\section{数据泄露漏洞}

\section{更安全的系统}
\subsection{SEAndroid}
NSA在1月16日发布了基于SELinux的SEAndroid\footnote{\url{http://selinuxproject.org/page/SEAndroid}}\index{SEAndroid}。

其主要工作和思想如下:

\begin{itemize}
  \item 要把SELinux系统及其设计思想移植到Android有不少困难,比如非标准Linux机制(Ashmem、Binder等模块)、Android框架层的策略、现有DAC权限控制模型、文件系统不支持安全标签等等,但SEAndroid项目最后解决了这些问题
  \item 对目前Android系统上发现的几乎所有提权漏洞、Android应用上发现的大量信息泄露漏洞,SEAndroid都能补上,主要是漏洞利用程序的利用步骤会被SEAndroid的安全策略所阻止,也就是无法顺利利用漏洞。
  \item 作者认为对Android,MAC模型是有意义的,并且可以和现有安全体系并存。
\end{itemize}

从RageAgainstTheCage、zergRush(NSA称之为GingerBreak)这两个漏洞利用的阻止细节上,SEAndroid的安全做得比较理想,对权限的管理已经到了及其严格的地步,利用代码几乎寸步难行。

而这个项目目前存在什么问题,NSA公开的资料中只字不提。笔者认为,对性能的损耗可能是第一位的。

在去年的一篇论文中有提到,即便是在Android所有敏感API的入口处基于源码做hook,记录API触发行为和参数,编译得到的系统的性能都有显著的降低,而这个降低的性能在真实机器上可能引发连锁效应,导致整个系统无法启动。

而SEAndroid从文件系统开始广泛使用标签,对每个需要管理的对象使用标签来实施安全策略,这里还包括对每个敏感行为的审计。笔者倾向于认为这样的改动也会遇到同样的问题。至于NSA有没有解决,是如何解决的,还需要进一步看细节。

\subsection{TainDroid}

\subsection{AppFence}
